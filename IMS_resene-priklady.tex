% !TEX TS-program = pdflatex
% !TEX encoding = UTF-8 Unicode

\documentclass[11pt]{article} % use larger type; default would be 10pt

\usepackage[utf8]{inputenc} % set input encoding (not needed with XeLaTeX)
\usepackage[czech]{babel}

%%% Examples of Article customizations
% These packages are optional, depending whether you want the features they provide.
% See the LaTeX Companion or other references for full information.

%%% PAGE DIMENSIONS
\usepackage{geometry} % to change the page dimensions
\geometry{a4paper} % or letterpaper (US) or a5paper or....
% \geometry{margin=1.5in} % for example, change the margins to 2 inches all round
% \geometry{landscape} % set up the page for landscape
%   read geometry.pdf for detailed page layout information

\usepackage{graphicx} % support the \includegraphics command and options

% \usepackage[parfill]{parskip} % Activate to begin paragraphs with an empty line rather than an indent

%%% PACKAGES
\usepackage{booktabs} % for much better looking tables
\usepackage{array} % for better arrays (eg matrices) in maths
\usepackage{paralist} % very flexible & customisable lists (eg. enumerate/itemize, etc.)
\usepackage{verbatim} % adds environment for commenting out blocks of text & for better verbatim
\usepackage{subfig} % make it possible to include more than one captioned figure/table in a single float
% These packages are all incorporated in the memoir class to one degree or another...
\usepackage{listings}
\usepackage{paralist}
\usepackage{amsmath}

%%% HEADERS & FOOTERS
\usepackage{fancyhdr} % This should be set AFTER setting up the page geometry
\pagestyle{fancy} % options: empty , plain , fancy
\renewcommand{\headrulewidth}{0pt} % customise the layout...
\lhead{}\chead{}\rhead{}
\lfoot{}\cfoot{\thepage}\rfoot{}

%%% SECTION TITLE APPEARANCE
\usepackage{sectsty}
\allsectionsfont{\sffamily\mdseries\upshape} % (See the fntguide.pdf for font help)
% (This matches ConTeXt defaults)

%%% ToC (table of contents) APPEARANCE
\usepackage[nottoc,notlof,notlot]{tocbibind} % Put the bibliography in the ToC
\usepackage[titles,subfigure]{tocloft} % Alter the style of the Table of Contents
\renewcommand{\cftsecfont}{\rmfamily\mdseries\upshape}
\renewcommand{\cftsecpagefont}{\rmfamily\mdseries\upshape} % No bold!

%%% END Article customizations


\lstset{
  language=C,                % choose the language of the code
  numbers=left,                   % where to put the line-numbers
  stepnumber=1,                   % the step between two line-numbers.        
  numbersep=5pt,                  % how far the line-numbers are from the code
  showspaces=false,               % show spaces adding particular underscores
  showstringspaces=false,         % underline spaces within strings
  showtabs=false,                 % show tabs within strings adding particular underscores
  tabsize=2,                      % sets default tabsize to 2 spaces
  captionpos=b,                   % sets the caption-position to bottom
  breaklines=true,                % sets automatic line breaking
  breakatwhitespace=true,         % sets if automatic breaks should only happen at whitespace
}


%%% The "real" document content comes below...

\title{IMS\\Příklady na semestrálku}
\author{Přebráno z Fitušky (fórum a wiki)}
%\date{} % Activate to display a given date or no date (if empty),
         % otherwise the current date is printed 

\begin{document}
\maketitle

%%%%%%%%%%%%%%%%%%%%%%%%%%%%%%%%%%%%%%%%%%%%%%%%%%%%%%%%%%%%%%%%%

\section{Programování na papír}



\subsection{Monte Carlo (nick)}

Napište v jazyku C funkci \texttt{double MCIntegral()} pro výpočet N-násobného určitého integrálu funkce \texttt{double f(double x[], unsigned N)} (podle všech X i , i náleží {0,...,N-1}) metodou Monte Carlo (integrál musí být N-násobný, jinak 0 bodů) s přesností přibližně 1\%. Meze integrálu jsou uloženy v globálním poli struktur \texttt{Range} s položkami \texttt{.min} a~\texttt{.max}. Předpokládejte, že neznáte rozsah funkčních hodnot funkce \texttt{f}. Máte k dispozici pouze funkci \texttt{double Random(void)} vracející pseudonáhodná čísla v rozsahu  $\langle 0,1)$, \texttt{f} a~žádnou jinou

\lstinputlisting{Monte.c}



\subsection{Kongruentní generátor pseudonáhodných čísel (nick)}
V ISO C napište funkci double \texttt{Random(void)} ve které bude váš vlastní kód kongruentního generátoru s rozsahem $\langle 0,1)$, konstanty \texttt{A}, \texttt{B}, \texttt{M} máte k dispozici. Máte zadánu funkci hustoty pravděpodobnosti $f(x)$ (nakreslete ilustrační obrázek).

\begin{align}
&f(x) = \frac{x}{2} \text{ pro x na intervalu $\langle0,2$)} \\
&f(x) = 0 \text{ jinak}
\end{align}

\begin{enumerate}[\itshape a\upshape)]
\item Napište funkci double \texttt{RandGenInv(void)}, která generuje číslo se zadaným rozložením s využitím metody inverzní transformace (jinak -5 bodů).
\item Napište funkci double \texttt{RandGenReject(void)}, která generuje číslo se zadaným rozložením s využitím metody vylučovací (jinak -5 bodů).
\item Vypočtěte pravděpodobnost, že náhodná proměnná $X$ s hustotou pravděpodobnosti $f(x)$ zadanou výše bude mít hodnotu $0.5$.
\item Vyjádřete distribuční funkci obecného spojitého rozložení $F(x)$ pomocí její funkce hustoty pravděpodobnosti $f(x)$.
\end{enumerate}

\lstinputlisting{Random.c}

\subsection*{Řešení}

\begin{enumerate}[\itshape a\upshape)]
\item
Rovnice
\begin{align}
&F(x) = \int \frac{x}{2} dx = \frac{x^2}{4} \\
&F^{-1}(x) = \sqrt{4x} = 2 \sqrt{x}
\end{align}

Program
\lstinputlisting{RandGenInv.c}

\item

Program
\lstinputlisting{RandGenReject.c}

\item
$$\int_{0.5}^{0.5} f(x) dx = 0$$
Poznámka: Pravděpodobnost jednoho konkrétního čísla z nekonečna (na spojitém intervalu je nekonečné množství reálných čísel) je vždy nulová. Aby byla nenulová, musí to být hodnota z intervalu (pak se nebudou meze integrálu rovnat).

\item
$$F(x) = \int_{-\infty}^{x}f(t)dt$$
Poznámka: $t$ je pomocná proměnná zavedená pro odlišení $x$, které je parametrem funkce $f(x)$ a podle kterého se derivuje, od $x$, které je vstupním parametrem funkce $F(x)$. Nesouvisí s žádnou proměnnou mimo tento výraz.
\end{enumerate}


\subsection{Algoritmy řízení simulace}
\subsubsection{Řízení diskrétní simulace -- Next-event (skeWer)}
\lstinputlisting{RizeniDiskretni.c}

\subsubsection{Řízení spojité simulace (Royce)}
\lstinputlisting{RizeniSpojita.c}

\subsubsection{Řízení kombinované simulace (skeWer)}
\lstinputlisting{RizeniKombinovana.c}


%%%%%%%%%%%%%%%%%%%%%%%%%%%%%%%%%%%%%%%%%%%%%%%%%%%%%%%%%%%%%%%%%

\newpage
\section{Markovské procesy}

\subsection{D/D/x, kalendáře}

\subsubsection{D/D/x bez poruchy (wiki)}
D/D/2 popsat stav simulace, kalendář, pořadí událostí a frontu v čase $6.5s$, začíná se v~$0s$, generuje se každou $1s$ a~zpracování trvá $3s$.
\vskip 0.25cm

\noindent
Časový diagram:
\vskip 0.1cm

\begin{tabular}{|c||ccccccc|cc|}
\hline
t & 0 & 1 & 2 & 3 & 4 & 5 & 6 & 7 & 8 \\
\hline
\hline
Z0 & P0 & P0 & P0 & P2 & P2 & P2 & P4 & P4  & P4\\
Z1 & & P1 & P1 & P1 & P3 & P3 & P3 & &\\
\hline
Q & & & P2 & P3 & P4 & P4 & P5 & & \\
 & & & & & & P5 & P6 & & \\
\hline
\end{tabular}

\vskip 0.3cm
\noindent
Seznam událostí:
\begin{enumerate}
\setcounter{enumi}{-1}
\item
 vygenerování P0, P0 obsazuje Z0
\item
 vygenerování P1, P1 obsazuje Z1
\item
 vygenerování P2, P2 jde do fronty
\item
 konec obsluhy P0, P2 obsazuje Z0, vygenerování P3, P3 jde do fronty
\item
 konec obsluhy P1, P3 obsazuje Z1, vygenerování P4, P4 jde do fronty
\item
 vygenerování P5, P5 jde do fronty
\item
 konec obsluhy P2, P4 obsazuje Z0, vygenerování P6, P6 jde do fronty
\\\line(1,0){300}
\\(dále jsou již jenom naplánované události)
\item
 konec obsluhy P3,  vygenerování P7
\item
\item
 konec obsluhy P4
\end{enumerate}

\vskip 0.3cm
\noindent
Fronta:
\begin{tabular}{|rl|}
\hline
$\rightarrow$ & P5 \\
\hline
& P6\\
\hline
\end{tabular}

\vskip 0.3cm
\noindent
Obsah kalendáře
\begin{tabular}{|c|c|l|}
\hline
čas & priorita & událost \\
\hline
7 & 0 & P3$\rightarrow$Release\\
7 & 0 & Generator$\rightarrow$Activate\\
9 & 0 & P4$\rightarrow$Release\\
\hline
\end{tabular}

\newpage

%%%%%%%%%%%%%%%%%%%%%%%%%

\subsubsection{D/D/x s poruchou (mesma)}
D/D/3, fronta LIFO, příchody každou 1s, obsluha trvá 3s. V čase 2s přijde porucha do Z0 a trvá 2s. Vyjádřit v čase $4.5s$.
\vskip 0.25cm

\noindent
Časový diagram:
\vskip 0.1cm

\begin{tabular}{|c||ccccc|cc|}
\hline
t & 0 & 1 & 2 & 3 & 4 & 5 & 6 \\
\hline
\hline
Z0 & P0 & P0 & x & x & P0 & & \\
Z1 & & P1 & P1 & P1 & P3 & P3 & P3\\
Z2 & & & P2 & P2 & P2 & & \\
\hline
Q & & & & P3 & P4 & & \\
\hline
\end{tabular}

\vskip 0.3cm
\noindent
Seznam událostí:
\begin{enumerate}
\setcounter{enumi}{-1}
\item
 vygenerování P0, P0 obsazuje Z0
\item
 vygenerování P1, P1 obsazuje Z1
\item
 porucha obsazuje Z0, vygenerování P2, P2 obsazuje Z2
\item
 vygenerování P3, P3 jde do fronty
\item
 konec poruchy Z0, konec obsluhy P1, P3 obsazuje Z1, vygenerování P4, P4 jde do fronty
\\\line(1,0){300}
\\(dále jsou již jenom naplánované události)

\item
 konec obsluhy P2, konec obsluhy P0, vygenerování P5
\item

\item
 konec obsluhy P3
\end{enumerate}

\vskip 0.3cm
\noindent
Fronta:
\begin{tabular}{|rl|}
\hline
$\rightarrow$ & P4 \\
\hline
\end{tabular}

\vskip 0.3cm
\noindent
Obsah kalendáře
\begin{tabular}{|c|c|l|}
\hline
čas & priorita & událost \\
\hline
5 & 0 & P2$\rightarrow$Release\\
5 & 0 & P0$\rightarrow$Release\\
5 & 0 & Generator$\rightarrow$Activate\\
7 & 0 & P3$\rightarrow$Release\\
\hline
\end{tabular}

%%%%%%%%%%%%%%%%%%%%%%%%%%%%%%%%%%%%%%%%%%%%%%%%%%%%%%%%%%%%%%%%%

\newpage
\section{Teorie (\textit{wiki})}
\begin{description}
\item[Abstraktní model] zjednodušený pospis zkoumaného systému
\item[Buňka (cell)] základní element, může být v jednom z konečného počtu stavů (0/1)
\item[Celulární automat] je souhrné označení pro určitý typ fyzikálního modelu reálné situace, slouží k časové i prostorové diskrétní ideální modelaci fyzikálních systémů, kde hodnoty veličin nabývají pouze diskrétních hodnot
\item[Chování systému] každému časovému průběhu vstupní veličiny přiřazuje časový průběh výstupních veličin, je dáno vzájemnými interakcemi mezy prvky systému a mezi systémem a jeho okolím
\item[Diskrétní simulace] simulace s dikrétními prvky a časem. Experimentování se simulačním modelem.
\item[Markovova vlastnost] následující stav procesu závisí jen na aktuálním stavu (ne na minulosti)
\item[Markovský proces] náhodný diskrétní proces se spojitým časem, který splňuje markovovu vlastnost
\item[Markovův řetězec] náhodný diskrétní proces s diskrétním časem, který splňuje markovovu vlastnost, ekvivalentem je konečný automat s pravděpodobnostmi přechodů
\item[Model] napodobenina systému jiným systémem
\item[Modelování] proces vytváření abstraktního modelu
\item[Modelový čas] časová osa modelu, může být reálný, zrychlený, zpomalený
\item[Monte Carlo] metoda používaná k výpočtu určitých integrálů, určená pro simulaci systémů, které jsou založeny na experimentování se stochastickými modely s využitím generátorů náhodných čísel
\item[Okolí (neighbourhood)] několik typů, liší se počtem okolních buněk, se kterými se pracuje
\item[Pole buněk (lattice)] n--rozměrné, obvykle 1D nebo 2D, rovnoměrné dělení prostoru, může být konečné nebo nekonečné
\item[Pravidla (rules)] funkce stavu buňky a jejího okolí definují nový stav buňky v čase s(t + 1) = f(s(t), Ns(t))
\item[Petriho síť] graf popisující stavy a přechody mezi stavy, umožňuje znázornění paralelismu a nedeterminismu
\item[Priorita obsluhy] proces s vyšší prioritou obsluhy než současný může vyhodit současný proces a sám obsadit zařízení
\item[Priorita procesu] pro řazení ve vnější frontě u zařízení a v kalendáři (primárne podľa času a sekundárne podľa priority)
\item[Proces] posloupnost událostí
\item[Reálný čas] probíhá v něm skutečný děj v reálném systému
\item[Řád metody numerické integrace] stupeň polynomu N (Taylorův rozvoj) definuje tzv. řád metody, tímto polynomem aproximujeme na základě aktuální hodnoty
\item[Rychlá smyčka] tvoří cyklus v orientovaném grafu závislostí funkčních bloků.
\item[SHO] systémy obsluhující zařízení, která poskytují obsluhu transakcím, součástí SHO jsou transakce, fronty, obslužné linky
\item[Simulace] experimentování s modelem, získávání nových znalostí
\item[Simulační model] abstraktní model zapsaný kódem
\item[Spojitý systém] všechny prvky systému mají spojité chování (nemění své parametry skokově)
\item[Statistiky] vytížení zařízení, délky front, doby čekání ve frontách, využití kapacity skladů, celková doba transakce strávené v systému, maximum, minimum, střední hodnoty, směrodatná odchylka, rozptyl, ...
\item[Stavová podmínka] diskrétní podmínka ve spojité simulaci, která může nějakým způsobem ovlivnit model (jiné rovnice pro výpočet kapaliny a páry), je citlivá na hranu, takže se vykonává pouze při změně, na základě její změny vyvolám stavovou událost
\item[Stavová událost] akce, která je podmíněna změnou pravdivostní hodnoty stavové podmínky (např.: otočení hodnoty integrátoru), diskrétní skoková změna systému
\item[Strojový čas] čas CPU spotřebovaný na výpočet programu (závisí na složitosti simulačního modelu, nesouvisí přímo s modelovým časem)
\item[Systém] soubor elementárních částí, které mají mezi sebou vazby
\item[Třídy statistik] stat, tstat (bere v potaz čas), histogram, mají společné operace s.Clear() (inicializace), s.Output() (tisk), s(x) (záznam hodnoty x)
\item[Událost] změna stavu diskrétního systému, jednorázová, nepřerušitelná
\item[Validace modelu] ověřování platnosti modelu a proces, kdy se snažíme dokázat, že pracujeme s modelem adekvátním modelovanému systému
\item[Verifikace modelu] dokazování izomorfnosti (rovnosti) abstraktního modelu s modelem simulačním, jinak řečeno ověřování korespondence abstraktního modelu se simulačním modelem

\end{description}

\end{document}
